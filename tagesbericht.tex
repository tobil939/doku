% ==================== Anfang des Dokuments
% Vorlage für einen kurzen Bericht um die Vorkomnisse an einen Messtag zu dokumentieren.
% Am besten vor Feierabend kurz ein paar Minuten zeit nehmen und das wichtigest eintragen.
% Keine Details, Adressen, Messdaten oder Anleitungen in diesesm Bericht dokumentieren.
% Zumindest solange der Ort der Kompilierung noch fragwürdig ist.

\documentclass[9pt,twoside]{article}                  % Artikel, beidseitig, Schriftgröße 9 

% ==================== Verwendete Pakete
\usepackage[utf8]{inputenc}                           % Umlaute Teil I 
\usepackage[T1]{fontenc}                              % Umlaute Teil II 
\usepackage[ngerman]{babel}                           % deutsches Sprachpacket 
\usepackage{fancyhdr}                                 % Kopf- und Fußzeile 
\usepackage{fancybox}                                 % Einfache Rahmen 
\usepackage{geometry}                                 % Seitenränder 
\usepackage{array}                                    % Bessere Darstellung von Tabellen 
%\usepackage{graphicx}                                 % Einfügen von Bildern und Grafiken 
\usepackage{setspace}                                 % Zeilenabstand 
\usepackage{amsmath}                                  % Mathematische Symbole 
\usepackage{textgreek}                                % Griechisches Alphabet 

% ==================== Seitenlayout und Seitenränder
\geometry{
  a4paper,                                            % Papierformat  
  left=1.5cm,                                         % Seitenränder 
  right=1.5cm,
  top=2cm,
  bottom=2cm
}
\setstretch{1.1}                                      % Zeilenabstand 

% ==================== Kopf- und Fußzeile
\pagestyle{fancy}                                     % Aufteilung von fancy verwenden 
\fancyhf{}                                            % Leere Kopf- und Fußzeilen Einfügen 
\setlength{\headheight}{15pt}                         % Platz für Kopfzeile 
\setlength{\footskip}{20pt}                           % Platz für Fußzeile 

% ==================== Kopfzeile
\fancyhead[L]{\textbf{Titel des Dokuments}}           % Titel oben Links 
\fancyhead[R]{\today}                                 % Datum des speicherns einfügen 

% ==================== Fußzeile
\fancyfoot[L]{Autor}                                  % Autor unten links 
\fancyfoot[C]{Abteilung}                              % Abteilungn unten mittig 
\fancyfoot[R]{Projekt}                                % Projekt unten rechts 

% ==================== Begin des Dokuments
\begin{document}

% ==================== Kurzfassung
% Kurze Zusammenfassung des Tages, ca. zwei Zeilen
\section*{Kurzfassung}
\begin{Sbox}                                          % erzeugt einen Rahmen um den Text 
  \begin{minipage}{\textwidth}                        % lässt den Rahmen um den Text wachsen 
    Bunga Bunga hart \\
  \end{minipage}
\end{Sbox}
\fbox{\TheSbox}                                       % einfacher Rahmen 
\vspace{1cm}                                          % erzeugt einen Abstand von 1cm 

% ==================== Wichtes 
% Besonderheiten und Sachen die herausgestochen sind
\section*{Besonderheiten / Wichtigste}
\begin{Sbox}
  \begin{minipage}{\textwidth}
    Bunga bunga hart \\
  \end{minipage}
\end{Sbox}
\doublebox{\TheSbox}                                  % doppelter Rahmen 
\vspace{1cm}

% ==================== Bericht
% Eigentlicher Bericht
\section*{Bericht}
\begin{Sbox}
  \begin{minipage}{\textwidth}
    bunga bunga hart \\ 
  \end{minipage}
\end{Sbox}
\fbox{\TheSbox}
\vspace{1cm}

\newpage                                              % springt auf eine neue Seite 
% ==================== Aufgaben 
% Aufgaben für einen anderen Zeitpunkt oder für jemand anderes 
\section*{Aufgaben}
\begin{Sbox}
  \begin{minipage}{\textwidth}
    Bunga bunga hart \\
  \end{minipage}
\end{Sbox}
\doublebox{\TheSbox}

% ==================== Ende des Dokuments 
\end{document}                                        % Bis hier geht die Kompilierung 




% ==================== Vorlagen
% vorlagen für haufig verwendete Anwendungen und Formate

% ==================== Textformen
\noindent                                             % Unterdrückt die Indentierung 
Textformen \\
\textbf{fett} \\ 
\textit{kursiv} \\ 
\underline{unterstrichen} \\ 
\underline{\underline{doppelt unterstrichen}} \\      % so können auch anderer Formate kombiniert werden 

% ==================== Zeichen
\noindent
Zeichen \\ 
\textmu \\                                            % verwendet das textgreek-Paket 
$ 47 \mu F $ \\                                       % verwendet das Mathe-Paket, inline Modus 
\textSigma \\ 
$ \sum $ \\                                           % Text im Mathemodus ist kursiv 
\textOmega \\ 
R = 22m\textOmega \\ 
$ R = 22m \Omega $ \\ 
$ \rightarrow  \uparrow  \downarrow  \leftarrow $ \\ 

% ==================== Aufzählungen
\noindent
Aufzählungen
\begin{itemize}
  \item Erster Punkt
  \item Zweiter Punkt
\end{itemize}

% ==================== Aufzählungen mit Nummern 
\noindent
Numerierungen
\begin{enumerate}
  \item Erster Punkt 
  \item Zweiter Punkt
\end{enumerate}

% ==================== Tabular
\noindent
Tabelle ohne Striche
\begin{tabbing}
  \hspace{2cm} \= \hspace{2cm} \= \hspace{4cm} \= \kill % Vorgabe der Spalten
  Spalte 1 \> Spalte 2 \> Spalte 3 \\                 % Zeilen hinzufügen wie man möchte 
  Wert 1 \> Wert 2 \> Wert 3 \\ 
\end{tabbing}

% ==================== Tabellen
\noindent
Tabelle mit Strichen 
\begin{tabular}{|c||c|c|} \hline                      % Definition der Tabelle | für Stirche  
  Spalte 1 & Spalte 2 & Spalte 3 \\ \hline \hline
  Wert 1 & Wert 2 & Wert 3 \\ \hline
\end{tabular}

% ==================== Bilder einfügen
Bild einfügen \\ 
\begin{figure}[h] 
  \centering                                          % Bild wird zentriert 
  \includegraphics[width=0.5\textwidth]{bild.jpg}     % Bild wird um 50% geschrumpft 
  \caption{Bildname}                                  % Name des Bildes, falls benötigt 
  \label{fig:beispiel}                                % Label, falls darauf verwiesen werden solle 
\end{figure}
Verweis auf das Bild im Text durch \ref{fig:beispiel}

% ==================== Boxen nebeneinander
\noindent
Zwei Boxen nebeneinander
\begin{Sbox}
  \begin{minipage}[t]{0.45\textwidth}                 % 45% des verfügbaren Platzes, t verstehe ich noch nicht 
    bunga bunga hart \\ 
  \end{minipage}
\end{Sbox}
\fbox{\TheSbox}
\begin{Sbox}
  \begin{minipage}[t]{0.45\textwidth}
    bunga bunga hart \\ 
  \end{minipage}
\end{Sbox}
\fbox{\TheSbox}
