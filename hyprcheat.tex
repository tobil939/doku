\documentclass[10pt,a4paper,oneside]{article}
\usepackage[utf8]{inputenc}
\usepackage{arev}
\usepackage[T1]{fontenc}
\usepackage{amsmath}
\usepackage{amsfonts}
\usepackage{amssymb}
\usepackage{makeidx}
\usepackage[colorlinks,
pdfpagelabels,
pdfstartview = FitH,
bookmarksopen = true,
bookmarksnumbered = true,
linkcolor = black,
plainpages = false,
hypertexnames = false,
citecolor = black] {hyperref}
\author{Tobi L}
\title{Hyprland Cheat Sheet}
\begin{document}
\section*{Tastenkombinationen}
\subsection*{Programme}
\begin{tabbing}
  \hspace*{1mm} \=\hspace{30mm} \=\kill
  \> alt+enter \> öffnet das Terminal \\ 
  \> alt+i \> öffnet den Dateimanager \\ 
  \> alt+d \> öffnet das Menü \\ 
  \> alt+z \> öffnet den PDF Viewer \\ 
  \> alt+u \> öffnet den Firefox \\ 
  \> alt+o \> öffnet Neovim /home/user Verzeichnis \\ 
  \> alt+p \> öffnet den Kalender \\ 
  \> alt+ü \> öffnet eine Email Programm \\ 
  \> alt+ö \> öffnet Audieinstellungen \\ 
  \> alt+ä \> öffnet Netzwerkeinstellungen \\ 
  \> alt++ \> öffnet die hyprland.conf in nvim \\ 
  \> alt+y \> macht ein Screenshot von einem Monitor \\ 
  \> alt+x \> macht ein Screenshot von einem Fenster \\
  \> alt+c \> macht ein Screenshot von einem Auswahlfenster \\
\end{tabbing}
\subsection*{Bedienung}
\begin{tabbing}
  \hspace*{1mm} \=\hspace{30mm} \=\kill
  \> alt+q \> schließt das ausgewählte Fenster \\ 
  \> alt+m \> schließt hyprland \\ 
  \> alt+v \> ausgewähltes Fenster kann überall hin verschoben werden \\ 
  \> alt+e \> wechselt die Teilung, horizontal/vertikal \\ 
  \> alt+shift+x \> sperrt den Bildschirm \\ 
  \> alt+1...0 \> öffnet oder wechselt zur Seite 1...0 \\
  \> alt+shift+1...0 \> verschiebt das Fenster zur Seite 1...0 \\
  \> alt+h \> wechselt nach links \\ 
  \> alt+j \> wechselt nach untern \\ 
  \> alt+k \> wechselt nach oben \\ 
  \> alt+l \> wechselt nach rechts \\ 
\end{tabbing}

\end{document}
