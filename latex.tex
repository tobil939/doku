\documentclass[10pt,a4paper,oneside]{report}
\usepackage[utf8]{inputenc}
\usepackage{arev}
\usepackage[T1]{fontenc}
\usepackage{amsmath}
\usepackage{amsfonts}
\usepackage{amssymb}
\usepackage{makeidx}
\usepackage[left=2cm,right=2cm,top=2cm,bottom=2cm]{geometry}
\usepackage[colorlinks,
pdfpagelabels,
pdfstartview = FitH,
bookmarksopen = true,
bookmarksnumbered = true,
linkcolor = black,
plainpages = false,
hypertexnames = false,
citecolor = black] {hyperref}
\author{Tobi L}
\title{LaTeX}
\begin{document}
\maketitle
\tableofcontents
\chapter{Installation}
Beziehe mich hier nur auf die Installation bei Arch, für andere Linux Distros, müsste pacman ersetzt werden durch z.B. apt-get install\\
\section{Compiler}
	\begin{verbatim}
		sudo pacman -S texlive
		sudo pacman -S texlive-utiliies
	\end{verbatim}
\section{Editoren}
\subsection{Texmaker}
Grafische Entwiklungsumgebung zum erstellen von LaTeX Dokumenten.\\
\subsubsection{Eigenschaften}
	\hspace*{5mm} - Assisten zum erstellen von Dokumenten\\
	\hspace*{5mm} - Anzeige der Dokumentenstruktur\\
	\hspace*{5mm} - Werkzeugleiste mit den wichtigsten LaTeX Befehlen\\
	\hspace*{5mm} - Buttons für Compilierung und Anzeige der PDF\\
	\hspace*{5mm} - Eingebaute Option das PDF direkt in der Anwendung anzeigen zu lassen\\
\subsubsection{Installation}
	\begin{verbatim}
		sudo pacman -S texmaker
	\end{verbatim}
\subsubsection{Konfiguration}
\begin{math} Options \rightarrow Configure\_Texmaker \rightarrow Commands \rightarrow PDF Viewer \end{math}\\
Um die PDF Datei im externen Fenster anzeigen zu lassen.\\
\subsection{Neovim}
Wenn man der Maus den Krieg erklären will, ambesten noch mit Arch und i3wm\\
\subsubsection{Eigenschaften}
	\hspace*{5mm} - \\
	\hspace*{5mm} - \\
	\hspace*{5mm} - \\
	\hspace*{5mm} - \\
	\hspace*{5mm} - \\
\subsubsection{Installation}
	\begin{verbatim}
		sudo pacman -S neovim
	\end{verbatim}
\subsubsection{Konfiguration}
lazyvim und plugin
\section{Konfiguration}
Aktuell fällt mir nichts ein. Hab bisher auch noch nichts konfiguriert.\\
War alles bisher nur Einstellungen in nvim.\\
\chapter{Bedienung}
\section{Compilieren}
Datei z.B. in nvim oder Texmaker  erstellen und als .tex speichern.\\	
\subsection{Compilieren}
Erstellt alle notwendigen Daten, wird empfohlen wenn der Text das erste Mal umgewandelt wird.\\
	\begin{verbatim}
		latex text.tex
	\end{verbatim}
\subsection{PDFLaTeX}
Erstellt die PDF-Datei, eine Art der schenll Umwandlung, verwende ich meist wenn ich den Text schon einmal
Umgewandelt hab.
	\begin{verbatim}
	 	pdflatex text.tex
	\end{verbatim}
\newpage
\section{Grundlagen für ein Dokument}
	\begin{verbatim}
		\documentclass[10pt,a4paper,twoside]{report}			
																			# Schriftgröße, Papierformat, beidseitig, Bericht
		\usepackage[utf8]{inputenc}						# verschiedene Pakete
		\usepackage[T1]{fontenc}									# vielleicht schreibe ich 
		\usepackage{amsmath}													# irgendwann mal
		\usepackage{amsfonts}												# mehr dazu
		\usepackage{amssymb}
		\usepackage{makeidx}
		\usepackage[colorlinks,										# wird benötigt um Links 
		pdfpagelabels,																			# im Inhaltsverzeichnis 
		pdfstartview = FitH,													# zu erstellen
		bookmarksopen = true,
		bookmarksnumbered = true,
		linkcolor = black,
		plainpages = false,
		hypertexnames = false,
		citecolor = black] {hyperref}	
		\author{Tobi L}																		# Autor
		\title{LaTeX}																				# Titel des Dokuments
		\begin{document}																	# Anfang des Dukuments
		\maketitle																							# Titelseite wird erstellt
		\tableofcontents																	# Inhaltsverzeichnis wird erstellt
		Dokument																									# hier zwischen das Dokument schreiben
		\end{document}																			# Ende des Dokuments
	\end{verbatim}
\section{Strukturen}
\begin{verbatim}
\part{}
\chapter{I}
\section{I.I}
\subsection{I.I.I}
\subsubsection{}
\paragraph{*}
\subparagraph{**}
\end{verbatim}
\section{Textformen}
\begin{verbatim}
\begin{flushleft}
Linksbündig
\end{flushleft}
\begin{center}
Zentriert
\end{center}
\begin{flushright}
Rechtsbündig
\end{flushright}
\end{verbatim}
\section{Schriftgröße}
\begin{tiny}
tiny
\end{tiny}
\begin{scriptsize}
scriptsize
\end{scriptsize}
\begin{footnotesize}
footnotesize
\end{footnotesize}
\begin{small}
small
\end{small}
\begin{normalsize}
normalsize
\end{normalsize}
\begin{large}
large
\end{large}
\begin{LARGE}
LARGE
\end{LARGE}
\begin{huge}
huge
\end{huge}
\begin{Huge}
Huge
\end{Huge}
\begin{verbatim}
\begin{tiny}
tiny
\end{tiny}
\begin{scriptsize}
scriptsize
\end{scriptsize}
\begin{footnotesize}
footnotesize
\end{footnotesize}
\begin{small}
small
\end{small}
\begin{normalsize}
normalsize
\end{normalsize}
\begin{large}
large
\end{large}
\begin{LARGE}
LARGE
\end{LARGE}
\begin{huge}
huge
\end{huge}
\begin{Huge}
Huge
\end{Huge}
\end{verbatim}
\section{Schriftakzente}
\textbf{Schriftakzente}  \textit{Schriftakzente}  \emph{Schriftakzente}\\
\underline{Schriftakzente} 
\begin{verbatim}
\textbf{Schriftakzente}  \textit{Schriftakzente}  \emph{Schriftakzente}
\underline{Schriftakzente} 
\end{verbatim}
\section{Leerzeichen}
\begin{verbatim}
\hspace{10mm}									# horizontaler Abstand 10mm
\hspace*{10mm}								# horizontaler Abstand 10mm vom linken Rand
\vspace{10mm}									# vertikaler Abstand 10mm
\vspace*{10mm}								# vertikaler Abstand vom oberen Rand
\\																				# neue Zeile
\newpage														# neue Seite
\end{verbatim}
\section{Symbole}
< >*+-=""'' \hspace{20mm} \# gehen ohne Befehle\\
\# \begin{verbatim} \#  \end{verbatim} 
\begin{math} \leq \end{math} \begin{verbatim} \begin{math} \leq \end{math}  \end{verbatim} 
\begin{math} \geq \end{math} \begin{verbatim} \begin{math} \geq \end{math}   \end{verbatim} 
\begin{math} \sim \end{math} \begin{verbatim}  \begin{math} \sim \end{math}  \end{verbatim} 
\begin{math} \leftarrow \end{math} \begin{verbatim}   \begin{math} \leftarrow \end{math} \end{verbatim} 
\begin{math} \Leftarrow \end{math} \begin{verbatim}   \begin{math} \Leftarrow \end{math}  \end{verbatim}
\begin{math} \rightarrow \end{math} \begin{verbatim}  \begin{math} \rightarrow \end{math}  \end{verbatim}
\begin{math} \Rightarrow \end{math} \begin{verbatim}  \begin{math} \Rightarrow \end{math} \end{verbatim}
\begin{math} \uparrow \end{math} \begin{verbatim}   \begin{math} \uparrow \end{math}  \end{verbatim}
\begin{math} \Uparrow \end{math} \begin{verbatim}  \begin{math} \Uparrow \end{math}   \end{verbatim}
\begin{math} \downarrow \end{math} \begin{verbatim}   \begin{math} \downarrow \end{math} \end{verbatim}
\begin{math} \Downarrow \end{math} \begin{verbatim}  \begin{math} \Downarrow \end{math}  \end{verbatim}
\begin{math} \leftrightarrow \end{math} \begin{verbatim}   \begin{math} \leftrightarrow \end{math}  \end{verbatim}
\begin{math} \updownarrow \end{math} \begin{verbatim} \begin{math} \updownarrow \end{math}   \end{verbatim}
\begin{math} /  ( ) \lbrace \rbrace \vert \Vert \end{math}
\begin{verbatim} \begin{math} /  ( ) \lbrace \rbrace \vert \Vert \end{math} \end{verbatim}
\section{Tabulator}
\begin{tabbing}
\hspace*{1mm} \=\hspace{30mm} \=\hspace{30mm}\= \kill
\> Befehl \> Bedeutung  \> Kommentar
\end{tabbing}
\begin{verbatim}
\begin{tabbing}
\hspace*{1mm} \=\hspace{30mm} \=\hspace{30mm}\= \kill
\> Befehl \> Bedeutung  \> Kommentar\\
\end{tabbing}
\end{verbatim}
\section{Befehle}
\textbf{\textit{math}} wird benötigt um Formel ein zu fügen oder manche Symbole\\
\begin{math}
5^{2}=\dfrac{1}{2}
\end{math}
\begin{verbatim}
\begin{math}
5^{2} \dfrac{1}{2}
\end{math}
\end{verbatim}
\vspace*{5mm}
\textbf{\textit{verbatim}} ermöglichtet es z.B. ganze Programmzeilen einzufügen
\begin{verbatim}
wert = eval(input("gib eine Zahl ein: "))
\end{verbatim}
\begin{verbatim}
\begin{verbatim}
wert = eval(input("gib eine Zahl ein: "))
end{verbatim}
\end{verbatim}
\section{Tabellen}
\subsection{Tabelle ohne Linien}
\begin{center}
\begin{tabular}[h]{lcccr}
  Spalte 1 & Spatlte 2 & Spalte 3 & Spalte 4 \\
  Zei1 2 1 & Zeile 2 2 & Zeile 2 3 & Zeile 2 4 \\
\end{tabular}
\end{center}
\begin{verbatim}
\begin{center}
\begin{tabular}[h]{lcccr}
  Spalte 1 & Spatlte 2 & Spalte 3 & Spalte 4 \\
  Zei1 2 1 & Zeile 2 2 & Zeile 2 3 & Zeile 2 4 \\
\end{tabular}
\end{center}
\end{verbatim}
\subsection{Tabelle mit Linien}
\begin{center}
\begin{tabular}[h]{|l|c|c|c|r|}
  \hline
  1 & 2 & 3 & 4 & 5 \\
  \hline 
  6 & 7 & 8 & 9 & 10 \\
  \hline 
  11 & 12 & 13 & 14 & 15 \\
  \hline 
  16 & 17 & 18 & 19 & 20 \\
  \hline 
  21 & 22 & 23 & 24 & 25 \\
  \hline
\end{tabular}
\end{center}
\begin{verbatim}
\begin{center}
\begin{tabular}[h]{|l|c|c|c|r|}
  \hline
  1 & 2 & 3 & 4 & 5 \\
  \hline 
  6 & 7 & 8 & 9 & 10 \\
  \hline 
  11 & 12 & 13 & 14 & 15 \\
  \hline 
  16 & 17 & 18 & 19 & 20 \\
  \hline 
  21 & 22 & 23 & 24 & 25 \\
  \hline
\end{tabular}
\end{center}
\end{verbatim}
\subsection{Tabellen Konfiguration}
\begin{verbatim}
\begin{tabular}[Platzierung]{Einteilung}
  Platzierung Eckigeklammern
  h here genau an der Stelle
  b boddom Seitenende
  t top Seitenanfang
  p positon Habe ich nicht weiter verfolgt

  Einteilung Geschweifteklammer
  c center Zentriert
  r right rechtsbündig
  l left linksbündig
  p(3mm) Spalte soll 3mm Breit sein
\end{verbatim}
\subsection{Beschriften und Verknüpfen}
\begin{table}[h]
\caption{Depp}
\begin{center}
\begin{tabular}[h]{|l|c|c|c|r|}
  \hline
  Spalte 1 & Spatlte 2 & Spalte 3 & Spalte 4 \\
  \hline 
  Zei1 2 1 & Zeile 2 2 & Zeile 2 3 & Zeile 2 4 \\
  \hline
\end{tabular}
  \label{tab:depp}
\end{center}
\end{table}
\begin{verbatim}
\begin{table}[h]
\caption{Depp}
\begin{center}
\begin{tabular}[h]{|l|c|c|c|r|}
  \hline
  Spalte 1 & Spatlte 2 & Spalte 3 & Spalte 4 \\
  \hline 
  Zei1 2 1 & Zeile 2 2 & Zeile 2 3 & Zeile 2 4 \\
  \hline
\end{tabular}
  \label{tab:depp}
\end{center}
\end{table}
\end{verbatim}
\begin{tabbing}
\hspace*{1mm} \=\hspace{30mm} \= \kill
\> caption \> erzeugt eine Überschrif\\t
\> \begin{math} label \lbrace tab:Name \rbrace \end{math} \> erzeugt eine Verknüpfung\\
\> \begin{math} ref \lbrace tab:Name \rbrace \end{math} \> erzeugt einen Link zur Tabelle\\
\end{tabbing}
\end{document}


