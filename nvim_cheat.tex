\documentclass[a4paper,11pt,twoside]{article}
\usepackage[ngerman]{babel}
\usepackage[utf8]{inputenc}
\usepackage[T1]{fontenc}
\usepackage[a4paper,margin=1.5cm]{geometry}
\pagestyle{empty}

\begin{document}
\section*{Modus}
\begin{tabbing}
  \hspace{3mm} \= \hspace{20mm} \= \kill
  \> i \> Einfügemodus vor Cursor \\
  \> a \> Einfügemodus nach Cursor \\
  \> r \> Zeichen unter Cursor ersetzen \\ 
  \> R \> überschreiben ab Cursor \\
  \> v \> markieren \\
  \> vb \> markiert Cursor bis Anfang vorheriges Word \\
  \> V \> ganze Zeilen markieren \\
  \> : \> Komandos \\
  \> ESC \> in den normaler Modus, auch aktueller Befehl abbrechen \\
\end{tabbing}

\section*{Bewegung}
\begin{tabbing}
  \hspace{3mm} \= \hspace{20mm} \= \kill
  \> h \> links \\ 
  \> j \> runter \\ 
  \> k \> hoch \\ 
  \> l \> links \\ 
  \> gg \> erste Zeile \\ 
  \> G \> letzte Zeile \\
  \> w \> Sprung Wortweise vorwärts \\ 
  \> b \> Sprung Wortweise rückwärts \\
  \> \^{} \> Sprung an Anfang der Zeile \\
  \> \$ \> Sprung an ende der Zeile \\
  \> :23 \> Sprung in die Zeile 23 \\ 
  \> 4j \> Sprung 4 Zeilen nach unten \\
  \> \# \> Sprung zum gleichen Wort oder Zeichen, rückwärts \\ 
  \> * \> Spring zum gleichen Wort oder Zeichen, vorwärts \\
  \> 23G \> Springt in Zeile 23 \\
  \> ( \> Springt einen Absatz nach oben, erstes Zeichen \\ 
  \> ) \> Springt einen Absatz nach unten, letztes Zeichen \\
  \> [ \> Springt einen Absatz nach oben \\ 
  \> ] \> Springt einen Absatz nach unten \\
\end{tabbing}

\section*{Ansicht}
\begin{tabbing}
  \hspace{3mm} \= \hspace{20mm} \= \kill
  \> zt \> Zeile mit Curosr oben zentrieren \\ 
  \> zz \> Zeile mit Cursor mittig zentrieren \\ 
  \> zb \> Zeile mit Cursor unten zentrieren \\
  \> Gzt \> Ende der Datei oben zentrieren \\ 
  \> 40Gzz \> Zeile Nr. 40 mittig zentrieren \\
\end{tabbing}

\section*{Löschen}
  \textbf{gelöschter Text wird meist ins Clipboard geschoben} \\ 
  \textbf{und überschreiben das letzte im Clipboard} \\
\begin{tabbing}
  \hspace{3mm} \= \hspace{20mm} \= \kill
  \> x \> löscht unter Cursor vorwärts \\
  \> X \> löscht unter Cursor rückwärts \\
  \> dd \> löscht ganze Zeile \\
  \> dw \> löscht Wort ab Cursor vorwärts \\
  \> db \> löscht Wort ab Cursor rückwärts \\
  \> di\{ \> löscht Inhalt von \{\} \\
\end{tabbing}

\section*{Ersetzen}
\begin{tabbing}
  \hspace{3mm} \= \hspace{20mm} \= \kill
  \> c \> ersetzen \\ 
  \> cw \> ersetzt unter Cursor vorwärts \\ 
  \> cb \> ersetzt unter Cursor rückwärts \\
  \> ci\{ \> ersetzt Inhalt von \{ \}
\end{tabbing}
\begin{verbatim} 
  :%s/bunga/bingo/gc
\end{verbatim}  
  erstzt bunga durch bingo, Nachfrage bei jedem Fall \\ 

\section*{Finden}
\begin{tabbing}
  \hspace{3mm} \= \hspace{40mm} \= \kill
  \> /bunga ENTER n \> sucht bunga und springt zum nächten bunga, N rückwärts \\ 
  \> ?bunga ENTER n \> sucht bunga und springt zum vorherigen bunga, N rückwärts \\
  \> g* \> sucht gleiches Wort unter Cursor, vorwärts \\
  \> g\# \> sucht gleiches Wort unter Cursor, rückwärts \\ 
  \> f\} \> sucht nach \}, F rückwärts \\
\end{tabbing}

\section*{Speichern}
\begin{tabbing}
  \hspace{3mm} \= \hspace{40mm} \= \kill
 \> :w \> speichern \\ 
 \> :w bunga.txt \> speichert unter bunga.txt \\ 
 \> :wa \> speichert alle Tabs \\ 
 \> :x \> speichert und öffnet neue Datei \\ 
 \> ZZ \> speichert und beendet \\ 
 \> :q! \> beenden ohen speichern \\ 
\end{tabbing}

\section*{Öffnen}
\begin{tabbing}
  \hspace{3mm} \= \hspace{40mm} \= \kill
 \> :e bunga.txt \> öffnet die Datei Bunga im gleichen Tab \\ 
 \> :tabnew bunga.txt \> öffnet Datei in neuem Tab \\
\end{tabbing}

\section*{Tabs}
\begin{tabbing}
  \hspace{3mm} \= \hspace{20mm} \= \kill
  \> :tabnew \> öffnet Leeren Tab \\ 
  \> gt \> sprint zum nächsten offenen Tab \\ 
  \> gT \> sprint zum vorherigen offenen Tab \\ 
  \> :tabs \> Liste mit offenen Tab \\ 
  \> :tabn 3 \> sprint zu Tab Nr. 3 \\ 
  \> :tabclose \> aktuellen Tab schließen \\ 
  \> :tabonly \> alle Tab bis auf aktuellen schließen \\
  \> :new \> neuen Tab horizontal \\ 
  \> :vnew \> neuen Tab vertiakal \\ 
  \> :close \> schließt aktuellen Tab \\ 
  \> :only \> schließt alle Tabs bis auf aktuellen \\ 
\end{tabbing}
\end{document}

%% Haupttext
%\section*{Einleitung}
%Hier beginnt dein Bericht. Du kannst mit der Einleitung starten und deine Gedanken klar strukturieren. Verwende Absätze, um die Lesbarkeit zu verbessern.
%
%\section*{Hauptteil}
%Im Hauptteil kannst du das Thema ausführlich behandeln. Gliedere deinen Text in Abschnitte und verwende Aufzählungen, falls nötig:
%
%\begin{itemize}
%    \item Erster Punkt
%    \item Zweiter Punkt
%    \item Dritter unkt
%\end{itemize}
%
%\section*{Schluss}
%Fasse im Schluss die wichtigsten Erkenntnisse deines Berichts zusammen und gib ggf. einen Ausblick.
%
%
